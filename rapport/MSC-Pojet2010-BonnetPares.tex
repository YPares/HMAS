\documentclass[a4paper,12pt]{report}
\usepackage[utf8]{inputenc}
\usepackage[T1]{fontenc}
\usepackage{lmodern}
\usepackage{listings}
\usepackage[french]{babel}
\usepackage{tabularx}
\usepackage{graphicx}
\usepackage{amsmath}
\usepackage{amsfonts}
\usepackage{textcomp}
\usepackage[hmargin=3.5cm, vmargin=3.5cm]{geometry}
\usepackage[colorlinks=true, linkcolor=black, urlcolor=black]{hyperref}


\title{Projet MSC : \\ Systèmes multi-agents hiérarchiques appliqués à la reconnaissance de formes}
\author{Bastien \textsc{Bonnet}}



\begin{document}

\pagenumbering{roman}

\maketitle

\pagenumbering{arabic}

\tableofcontents

\chapter{Introduction}
%\addcontentsline{toc}{chapter}{Introduction}

\section{Problématique}
L'objectif général de ce projet de Modélisation des Systèmes Complexes (MSC) est de définir et d'utiliser un Système Multi-Agents (SMA) hiérarchique. La problématique à laquelle nous avons tenté d'apporter une solution technologique au cours de ce projet est l'application d'un SMA hiérarchique à la reconnaissance de formes.

\section{Hypothèses de travail}
Nous nous proposons de reconnaître des formes de type chiffre. Plus précisément, nous sommes partis des hypothèses suivantes :
\begin{itemize}
 \item Les formes à reconnaître seront des chiffres de 0 à 9;
 \item Les formes seront stockées sous formes d'images d'une définition standardisée de 200 pixels (hauteur) sur 100 pixels (largeur);
 \item Les formes à reconnaître ont une hauteur supérieure à 70\% de la hauteur de l'image;
 \item Le style d'écriture est manuel, mais proche de celui utilisé par les radios-réveils (uniquement des barres tracées à la main).
\end{itemize}


\chapter{Conception}
\section{SMA hiérarchique : définition}
Un SMA hiérarchique est un SMA dans lequel les agents sont réparties en différents niveaux. Les agents de niveau 0, les plus bas dans la hiérarchie, ne supervisent personne. Tous les agents appartenant à un niveau supérieur à 0 supervisent plusieurs agents du niveau immédiatement en-dessous d'eux.

Un agent appartenant à un niveau de hiérarchie supérieur à 0 ne supervise pas obligatoirement la totalité des agents du niveau en-dessous de lui.

\section{La hiérarchie de notre SMA}
La hiérarchie de notre SMA est la suivante :
\begin{itemize}
 \item Niveau 0 : agents de type \emph{PixelAgent}
 \item Niveaux 1 à n-1 : agents de type \emph{HorizontalAgent} et \emph{VerticalAgent}
 \item Niveau n : un agent de type \emph{PatternAgent}
\end{itemize}

~

Les agents de type \emph{PixelAgent} sont des agents qui se fixent lorsqu'ils tom\-bent sur un pixel noir. Ils ont une taille de 1.

Les agents de type \emph{HorizontalAgent} et \emph{VerticalAgent} sont des agents qui vont se fixer en fontion du nombre d'agents de niveau inférieur fixés situés \textbf{complètement} en-dessous d'eux. Par exemple, un \emph{HorizontalAgent} de niveau 2 ne se fixera que s'il se trouve physiquement au-dessus de plus de 3 \emph{HorizontalAgent} fixés.

Enfin, les agents de type \emph{PatternAgent} sont des agents qui prendront une décision quant à la forme à reconnaître. Pour cela ils se baseront sur le nombre et le type (horizontal ou vertical) des agents du niveau en-dessous qui sont physiquement situés en-dessous de lui. Par exemple, un \emph{PatternAgent} reconnaîtra un 3 s'il trouve 3 \emph{HorizontalAgent} ainsi que 1 \emph{VerticalAgent}. 

Un \emph{PatternAgent} a une taille égale à celle de l'image contenant la forme à reconnaître. 

\section{Positionnement et déplacement des agents}
Les agents peuvent se déplacer par-dessus d'autres agents de même niveau hiérarchique, mais des agents de même niveau ne peuvent se fixer partiellement ou totalement l'un sur l'autre.

Actuellement, le déplacement de nos agents s'effectue de manière aléatoire.

\section{Architecture}
Un objet de classe \emph{World} contient les agents ainsi que la représentation du monde qui est basée sur l'image contenant la forme à reconnaître.

L'évolution du monde est basée sur le concept de tour : chaque tour, chacun de agents, s'il n'est pas fixé, tente de se déplacer et de se fixer.

\section{Diagramme de classes}
La figure \ref{fig:dc} présente le diagramme de classes du projet.
\begin{figure}
 \includegraphics[scale=.29, angle=90]{images/classes.pdf}
 \caption{Diagramme de classes}
 \label{fig:dc}
\end{figure}


\chapter{Développement}
\section{Choix des technologies}
Les technologies initialement retenues ont été le Python, Matlab et Java. Python et Java ont été retenus pour leur flexibilité. Matlab a été retenu car nous avions déjà effectué un projet de reconnaissance de formes avec.

Python a été éliminé car l'un des deux membres de l'équipe ne disposait que de connaissances de bases dans ce langage, et Matlab à cause de son manque de flexibilité (la programmation objet y est très limitée). Java a donc été notre choix de langage de développement.

Dans notre bibliographie, nous avions évoqué une API Java, \emph{MAGIQUE} (\textbf{M}ulti-\textbf{Agent} hiérarch\textbf{ique}), permettant de développer des SMA hiérarchique en Java. Cependant, après avoir étudié l'API, il s'est avéré que celle-ci était surdimensionnée pour nos besoins. En effet, nous ne nécessitions ni liaisons dynamiques entre agents, ni apprentissage dynamique des compétences, ce qui est le cœur de cette API. 

\section{Perspectives}
Avec plus de temps, il serait possible d'apporter des améliorations au systèmes. Tout d'abord, il serait possible de mieux placer les agents à leur création. Actuellement ils sont placés aléatoirement, mais il est certains qu'en les disposant de manière \og{}équitable\fg{} sur toute la surface de l'image, ils trouveraient plus vite un endroit où se fixer.

Ensuite, il serait possible de donner une direction de déplacement aux agents pour leur faire couvrir un maximum de surface.

De plus, actuellement, nous ne disposons que d'agents de type \og{}barre\fg{}, mais avec des agents capables de reconnaître d'autres formes, il serait possible d'être plus flexibles sur la calligraphie des formes à reconnaître.

Ces perspectivent sont facilement implémentables grâce à l'architecture de notre programme. En effet, la classe abstraite \emph{Agent} contient de nombreuses méthodes rendant aisé le développement de classes dérivées.

%Mieux gérer le chevauchement des agents longs

%Orienter direction agents

%Agents formes différentes

%tweak, tailles intermédiaires

\chapter{Conclusion}
%\addcontentsline{toc}{chapter}{Conclusion}
\section{Résultats}
Actuellement, du fait de nos critères basés sur la composition des chiffres par des barres horizontales et verticales, nous pouvons distinguer les chiffres suivants : 0, 1, 3, 4, 7, et la classe \{2, 5, 6, 8, 9\}. Par contre, du fait du déplacement aléatoire de nos agents, cela prend plus de 2 mn par forme à reconnaître en moyenne pour reconnaître :
\begin{itemize}
 \item les 0, 1, 4 avec 90\% de succès
 \item les 3 et 7 avec 70\% de succès
 \item classe \{2, 5, 6, 8, 9\} avec de 70\% de succès
\end{itemize}

~

Ces statistiques ont été obtenues de la manière suivante : 10 tentatives par imagette contenant un chiffre de 0 à 9. Pour disposer de statistiques vraiment significatives, il aurait été intéressant d'effectuer plus de tentatives par classe. Cependant, genérer les résultats présentés ci-dessus demande déjà plus de 3h. Les tests nous ont principalement servi à paramétrer nos agents (taille, nombre) et le nombre de tours à jouer dans le monde.


\section{Conclusion générale}
Ce projet nous a permis de nous approprier les notions de bases liées à un système multi-agents hiérarchique, et donc d'aller un peu au-delà du cours. De plus, son application à un problématique d'actualité, la reconnaissance de forme, nous  a permis de mesurer les possibilités concrètes offertes par les SMA.

\end{document}