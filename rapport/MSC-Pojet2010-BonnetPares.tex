\documentclass[a4paper,12pt]{report}
\usepackage[utf8]{inputenc}
\usepackage[T1]{fontenc}
\usepackage{lmodern}
\usepackage{listings}
\usepackage[french]{babel}
\usepackage{tabularx}
\usepackage{graphicx}
\usepackage{amsmath}
\usepackage{amsfonts}
\usepackage{textcomp}
\usepackage[hmargin=3.5cm, vmargin=3.5cm]{geometry}
\usepackage[colorlinks=true, linkcolor=black, urlcolor=black]{hyperref}


\title{Projet MSC : \\ Systèmes multi-agents hiérarchiques appliqués à la reconnaissance de formes}
\author{Bastien \textsc{Bonnet}}



\begin{document}

\pagenumbering{roman}

\maketitle

\pagenumbering{arabic}

\tableofcontents

\chapter{Introduction}
%\addcontentsline{toc}{chapter}{Introduction}

\section{Problématique}
L'objectif général de ce projet de Modélisation des Systèmes Complexes (MSC) est de définir et d'utiliser un Système Multi-Agents (SMA) hiérarchique. La problématique à laquelle nous avons tenté d'apporter une solution technologique au cours de ce projet est l'application d'un SMA hiérarchique à la reconnaissance de formes.

\section{Hypothèses de travail}
Nous nous proposons de reconnaître des formes de type chiffre. Plus précisément, nous sommes partis des hypothèses suivantes :
\begin{itemize}
 \item Les formes à reconnaître seront des chiffres de 0 à 9;
 \item Les formes seront stockées sous formes d'images d'une définition standardisée de 200 pixels (hauteur) sur 100 pixels (largeur);
 \item Les formes à reconnaître ont une hauteur supérieure à 70\% de la hauteur de l'image;
 \item Le style d'écriture est manuel, mais proche de celui utilisé par les radios-réveils (uniquement des barres tracées à la main).
\end{itemize}


\chapter{Conception détaillée}
\section{La hiérarchie du SMA}

\section{Diagramme de classes}
La figure \ref{fig:dc} présente le diagramme de classes du projet.
\begin{figure}
 %\includegraphics{}
 \caption{Diagramme de classes}
 \label{fig:dc}
\end{figure}


\chapter{Développement}
\section{Choix des technologies}
Les technologies initialement retenues ont été le Python, Matlab et Java. Python et Java ont été retenus pour leur flexibilité. Matlab a été retenu car nous avions déjà effectué un projet de reconnaissance de formes avec.

Python a été éliminé car l'un des deux membres de l'équipe ne disposait que de connaissances de bases dans ce langage, et Matlab à cause de son manque de flexibilité (la programmation objet y est très limitée). Java a donc été notre choix de langage de développement.

Dans notre bibliographie, nous avions évoqué une API Java, \emph{MAGIQUE} (\textbf{M}ulti-\textbf{Agent} hiérarch\textbf{ique}), permettant de développer des SMA hiérarchique en Java. Cependant, après avoir étudié l'API, il s'est avéré que celle-ci était surdimensionnée pour nos besoins. En effet, nous ne nécessitions ni liaisons dynamiques entre agents, ni apprentissage dynamique des compétences, ce qui est le cœur de cette API. 



\chapter{Conclusion}
%\addcontentsline{toc}{chapter}{Conclusion}
\end{document}