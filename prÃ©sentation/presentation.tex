\documentclass{beamer}

%\usepackage{beamerthemesplit}
\usepackage[francais]{babel}
\usepackage[utf8]{inputenc}
\usepackage[T1]{fontenc}
\usepackage{lmodern}
\usepackage{amsmath}
\usepackage{graphicx}
\usepackage{amssymb}
\usepackage{geometry}
\usepackage{listings}
\usepackage{color}
\usepackage{caption}
\usepackage{ulem}

\title{SMA Hiérarchiques\\Application à la reconnaissance de formes}
\author{Bastien \textsc{Bonnet}, Yves \textsc{Parès}}
%\date{2001-11-16}
\usetheme{Singapore}

\begin{document}

% \nocite{*}
% \bibliographystyle{unsrt}

\setbeamertemplate{footline}
{
  %\vspace{3mm}
  \begin{picture}(0,0)
    %\put(33,5){{\color{gray}\begin{minipage}{40mm} \insertauthor{} \end{minipage} }}
    %\put(-150,20){\begin{minipage}{\textwidth}\color{gray} \raggedleft \tiny \insertdate{} \end{minipage}}
    \put(-10,20){\begin{minipage}{\textwidth}\color{gray} \raggedleft \tiny \insertframenumber{} / \inserttotalframenumber{} \end{minipage}}
  \end{picture}
  
}

\setbeamertemplate{sidebar right}
{
}

\frame{
    \maketitle
}

%\section[Outline]{}
\frame{\tableofcontents}


\section{SMA Hiérarchique}

\frame {
  \frametitle{\insertsection}

  \begin{block}{Principes généraux}
  \begin{itemize}
    \item N niveaux d'agents,
    \item Tout agent $\notin{}niveau_N$ est supervisé,
    \item Communication entre agents limitée.
  \end{itemize}
  \end{block}
  
  \pause{}

  \begin{block}{Notre cas}
  \begin{itemize}
    \item Agents répartis en niveaux,
    \item Communication par observation uniquement,
    \item Niveau $N$ voit uniquement :
      \begin{itemize}
        \item Fils : Niveau $N-1$,
        \item Frères : Autres du niveau $N$.
      \end{itemize}
  \end{itemize}
  \end{block}
}


\section{Reconnaissance de formes}

\frame{
  \frametitle{\insertsection}
  
  \begin{block}{Formes et images simples}
  \begin{itemize}
    \item Chiffres style radio-réveil,
    \item Tracés à la main,
    \item Images binaires en 100x200.
  \end{itemize}
  \end{block}
}

\frame{
  \frametitle{Agents utilisés}

  \begin{block}{Agents}
  \begin{itemize}
    \item Taille,
    \item Position,
    \item Niveau,
    \item EstFixé,
    \item Méthode \textit{step}.
  \end{itemize}
  \end{block}
}

\frame{
  \frametitle{Mécaniques de reconnaissance}

  \begin{block}{Architecture de base}
  \begin{itemize}
    \item Image = tableau de booléens,
    \item Agent,
    \item World.
  \end{itemize}
  \end{block}

  \pause{}

  \begin{block}{Agents}
  3 types d'agents sur 5 niveaux :
  \begin{itemize}
    \item PixelAgent (niv. 0),
    \item Vertical et HorizontalAgent (niv. 1 à 3),
    \item PatternAgent (unique, niv. 4).
  \end{itemize}
  \end{block}
}

\section{Démonstration}
\frame {
  \centering \huge \insertsection  
}

\section{Conclusion}
\frame {
  \frametitle{\insertsection}
    Résultats :
    \begin{itemize}
 \item les 0, 1, 4 avec 90\% de succès
 \item les 3 et 7 avec 70\% de succès
 \item classe \{2, 5, 6, 8, 9\} avec de 70\% de succès
\end{itemize}


  \begin{itemize}
    \item Paramétrage empirique,
    \item Déplacements aléatoires,
    \item Très lent,
    \item Conception permet système paramétrable et évolutif.
  \end{itemize}
}

\end{document}

