\section{Liens consultés}

\begin{itemize}
  \item http://www.lifl.fr/SMAC/projects/magique/presentation/presentationContent.html
  \item \url{http://www.google.fr/url?sa=t&source=web&cd=1&ved=0CCAQFjAA&url=http%3A%2F%2Fciteseerx.ist.psu.edu%2Fviewdoc%2Fdownload%3Fdoi%3D10.1.1.48.8023%26rep%3Drep1%26type%3Dpdf&rct=j&q=A%20Hybrid%20and%20Hierarchical%20Multi-Agent%20Architecture%20Model&ei=G0oFTc2pNZCVswbqqfn0CQ&usg=AFQjCNEd9YSM8fR5ObG5K4gZnyRcsW738w&cad=rja}
  \item http://webloria.loria.fr/~vthomas/recherche/araignees
  \item http://webloria.loria.fr/~vthomas/fichiers/pdf/vthomas-wias02.pdf

  % Liens qu'on a laissé tomber :
  \item http://iis.ipipan.waw.pl/2009/proceedings/iis09-61.pdf
  \item http://hrcak.srce.hr/file/69374
  \item http://portal.acm.org/citation.cfm?id=1630553
  \item http://www.ccis2k.org/iajit/PDF/vol.5,no.3/10-202.pdf
  \item http://www.lifl.fr/~routier/recherche/publis/2000-06\_ps.zip
  \item http://www.lifl.fr/~routier/recherche/publis/TSI-tele\_ps.zip
\end{itemize}


% Pauchet a dit :
% Pour reco formes : approche Bottom-Up (Les agents de + haut niveau observent les agents de + bas niveau)

