\section{Un modèle d'architecture de \smahLong{s} : MAGIQUE}
\textbf{Articles résumés :} \cite{m1}, \cite{m2}

Deux articles sont ici résumés, car ils traitent du même sujet, se recoupent sur un certain nombre de points et ont un auteur en commun (P. Mathieu). Ces deux articles on été analysé et comparés, et le résumé de cette analyse se trouve ci-après.



Les auteurs de ces deux articles font partie du Laboratoire d'Informatique Fondamentale de Lille (LIFL) de l'Université de Lille 1.

~

Selon la vision des auteurs, un agent est une \og{}coquille vide\fg{} dans laquelle on greffe des compétences, qui sont donc clairement différenciées de celui-ci. Les compétences minimales incluses dans tout agent sont : la capacité de communiquer avec d'autres agents et celle d'apprendre de nouvelles compétences. Une compétence est composée d'un ensemble de fonctionnalités exploitables par un agent.

MAGIQUE est un framework Java de développement d'applications multi-agents, utilisant une approche \og{}orientée agent\fg{}. Dans MAGIQUE, l'organisation des agents est (à la base) hiérarchique, c'est-à-dire que chaque agent est récursivement constitué d'agents plus petits. une hiérarchie peut être vue comme un arbre dont la racine est un agent et dont les hypothétiques fils sont des hiérarchies. Les feuilles sont des agents spécialistes et les autres des superviseurs ; c'est dire que les superviseurs peuvent être répartis sur un nombre quelconque de niveaux hiérarchiques.

La hiérarchie représente les canaux de communications entre agents par défaut (communication verticale). Il est aussi possible de basculer d'un modèle hiérarchique strict à un modèle dit \og{}hybride\fg{} en introduisant des liens d'\og{}accointance\fg{} entre agents (communication horizontale), indépendants de la hiérarchie, dans le but de créer dynamiquement (c-à-d. après un certain d'exécution du SMA) un lien entre deux agents effectuant des requêtes fréquentes de l'un vers l'autre. Cela permet d'obtenir un comportement auto-adptatif du SMA et de diminuer les communications.

Dans MAGIQUE, l'architecture si situe entre le modèle autonome (les agents sont autonomes et il n'y a pas de structure commune dédiée au partage d'information) et le modèle à tableau noir (espace de recherche partagé entre agents où s'inscrivent les connaissances partagées, les hypothèses et les solutions formulées par les agents).


