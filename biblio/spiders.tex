\section{Détection de régions à base d'agents-araignées}

% Permet de détecter des ``régions à trous''

L'article \cite{spiders2} est issu des travaux de Vincent Thomas du Laboratoire Lorrain de Recherche en Informatique et ses Applications (LORIA) de l'université Nancy 2.
Il est très rapidement présenté sur sa page \cite{spiders1}.

L'article présente une méthode d'extraction de régions d'images par un ensemble d'araignées obéissant au principe de \emph{stigmergie} : le travail de tout agent est guidé par les traces laissées précédemment dans son environnement par lui-même ou par d'autres agents. C'est le cas des phéromones pour les colonies de fourmis ou bien, dans notre cas, les fils de soie pour les araignées sociales.

Tous les agents présents sont des agents réactifs : leurs actions sont déterminées par de simples stimulus. 

